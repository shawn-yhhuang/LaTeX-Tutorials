\documentclass{article}
% in order to include figures, we need to introduce some packages
\usepackage{graphicx}    % the old version is called "graphics"
\usepackage{subfig}        % allows subfigure environment
% \usepackage{graphicx, subfig}
\begin{document}
% table and figures are "environments" in LaTeX
% environments need a "begin" and "end"


                      % specify locations
\begin{figure}[!h]
\begin{center}    % center environment
% ! = overrides LaTeX preferences; h = here; t = top; b = bottom; p = special page for floats (floating objects)
% if leave the location blank, h is the default
\includegraphics[angle=45,width=.8\textwidth]{Stardew_Main_Logo}
\caption{A great and cheap game!}
% \includegraphics[angle,dimensions]{filename}
% it is okay to set dimensions like width=5in or width=10cm
% but better make is dynamic
\label{fig:logo}
% setting reference to this figure by create a label inside the figure environment.


%% subfloat is part of package "subfig"
%% \subfloat[caption]{\label{fig:XXX}
%% \includegraphics[angle,dimension]{filename}
%% }
%% using \quad to separate figures horizontally(?)
\subfloat[Logo]{\label{fig:horizon1}
\includegraphics[width=1\textwidth]{Horizon_logo}
}\\
\subfloat{\label{fig:horizon2}
\includegraphics[width=.5\textwidth]{horizon_zero_dawn_stormbird_guerrilla}
}
\subfloat{\label{fig:horizon3}
\includegraphics[width=.5\textwidth]{target_practice}
}
\caption{Horizon Zero Dawn}
\end{center}
\end{figure}

\section{Figure}
I have included a figure to explain, please see figure \ref{fig:logo}, \ref{fig:horizon1}





\end{document}